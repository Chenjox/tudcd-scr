\documentclass[a4paper,twoside,useseriffont=false]{tudcdartcl}
\usepackage[T1]{fontenc}
\usepackage[ngerman=ngerman-x-latest]{hyphsubst}

\usepackage[ngerman]{babel}
\usepackage[babel]{microtype}
\usepackage{blindtext}

%\TUDCDoption{paper}{a4paper}
\TUDCDoption{usemargin}{true}

\usepackage{multicol}
\usepackage{marginnote}

%\usepackage{tudcdfonts}

\begin{document}

\raggedright
\begin{multicols}{2}
    \blindtext
    \marginnote{\raggedright Hallo ich bin ein Text in der Marginalie}

    \blindtext[3]
    \marginnote{\raggedright Hallo ich bin ein Text in der Marginalie}

    \blindtext[4]
    \marginnote{\raggedright Hallo ich bin ein Text in der Marginalie}

    \blindtext[2]
    \marginnote{\raggedright Hallo ich bin ein Text in der Marginalie}

    \blindtext[5]
    \marginnote{\raggedright Hallo ich bin ein Text in der Marginalie}
\end{multicols}

\clearpage

Ich bin ein Pangramm

Stanleys Expeditionszug quer durch Afrika wird von jedermann bewundert.

\verb|\textbf|
\textbf{Stanleys Expeditionszug quer durch Afrika wird von jedermann bewundert.}

\verb|\textrm|
\textrm{Stanleys Expeditionszug quer durch Afrika wird von jedermann bewundert.}

\verb|\textsf|
\textsf{Stanleys Expeditionszug quer durch Afrika wird von jedermann bewundert.}

\verb|\texttt|
\texttt{Stanleys Expeditionszug quer durch Afrika wird von jedermann bewundert.}

\verb|\textsl|
\textsl{Stanleys Expeditionszug quer durch Afrika wird von jedermann bewundert.}

\verb|\textsc|
\textsc{Stanleys Expeditionszug quer durch Afrika wird von jedermann bewundert.}

\verb|\textup|
\textup{Stanleys Expeditionszug quer durch Afrika wird von jedermann bewundert.}

\verb|\textit|
\textit{Stanleys Expeditionszug quer durch Afrika wird von jedermann bewundert.}
%

        Stanleys \textbf{Expeditionszug} \textrm{quer} \textsf{durch} \texttt{Afrika} \textsl{wird} \textsc{von} \textup{jedermann} \textit{bewundert.}

\textbf{Stanleys \textbf{Expeditionszug} \textrm{quer} \textsf{durch} \texttt{Afrika} \textsl{wird} \textsc{von} \textup{jedermann} \textit{bewundert.}}

\textrm{Stanleys \textbf{Expeditionszug} \textrm{quer} \textsf{durch} \texttt{Afrika} \textsl{wird} \textsc{von} \textup{jedermann} \textit{bewundert.}}

\textsf{Stanleys \textbf{Expeditionszug} \textrm{quer} \textsf{durch} \texttt{Afrika} \textsl{wird} \textsc{von} \textup{jedermann} \textit{bewundert.}}

\texttt{Stanleys \textbf{Expeditionszug} \textrm{quer} \textsf{durch} \texttt{Afrika} \textsl{wird} \textsc{von} \textup{jedermann} \textit{bewundert.}}

\textsl{Stanleys \textbf{Expeditionszug} \textrm{quer} \textsf{durch} \texttt{Afrika} \textsl{wird} \textsc{von} \textup{jedermann} \textit{bewundert.}}

\textsc{Stanleys \textbf{Expeditionszug} \textrm{quer} \textsf{durch} \texttt{Afrika} \textsl{wird} \textsc{von} \textup{jedermann} \textit{bewundert.}}

\textup{Stanleys \textbf{Expeditionszug} \textrm{quer} \textsf{durch} \texttt{Afrika} \textsl{wird} \textsc{von} \textup{jedermann} \textit{bewundert.}}

\textit{Stanleys \textbf{Expeditionszug} \textrm{quer} \textsf{durch} \texttt{Afrika} \textsl{wird} \textsc{von} \textup{jedermann} \textit{bewundert.}}


Dieser Satz besteht aus acht A, sechs B, sechs C, sieben D, fünfundvierzig E, acht F, vier G, neun H,
fünfundzwanzig I, einem J, einem K, zwei L, elf M, achtundzwanzig N, einem O, einem P, einem Q, sieben R,
dreizehn S, sieben T, sieben U, fünf V, vier W, einem X, einem Y, zehn Z, einem Ä, einem Ö, vier Ü und einem ß.


\textbf{Dieser Satz besteht aus acht A, sechs B, sechs C, sieben D, fünfundvierzig E, acht F, vier G, neun H,
fünfundzwanzig I, einem J, einem K, zwei L, elf M, achtundzwanzig N, einem O, einem P, einem Q, sieben R,
dreizehn S, sieben T, sieben U, fünf V, vier W, einem X, einem Y, zehn Z, einem Ä, einem Ö, vier Ü und einem ß.}

\textrm{Dieser Satz besteht aus acht A, sechs B, sechs C, sieben D, fünfundvierzig E, acht F, vier G, neun H,
fünfundzwanzig I, einem J, einem K, zwei L, elf M, achtundzwanzig N, einem O, einem P, einem Q, sieben R,
dreizehn S, sieben T, sieben U, fünf V, vier W, einem X, einem Y, zehn Z, einem Ä, einem Ö, vier Ü und einem ß.}

\textsf{Dieser Satz besteht aus acht A, sechs B, sechs C, sieben D, fünfundvierzig E, acht F, vier G, neun H,
fünfundzwanzig I, einem J, einem K, zwei L, elf M, achtundzwanzig N, einem O, einem P, einem Q, sieben R,
dreizehn S, sieben T, sieben U, fünf V, vier W, einem X, einem Y, zehn Z, einem Ä, einem Ö, vier Ü und einem ß.}

\texttt{Dieser Satz besteht aus acht A, sechs B, sechs C, sieben D, fünfundvierzig E, acht F, vier G, neun H,
fünfundzwanzig I, einem J, einem K, zwei L, elf M, achtundzwanzig N, einem O, einem P, einem Q, sieben R,
dreizehn S, sieben T, sieben U, fünf V, vier W, einem X, einem Y, zehn Z, einem Ä, einem Ö, vier Ü und einem ß.}

\textsl{Dieser Satz besteht aus acht A, sechs B, sechs C, sieben D, fünfundvierzig E, acht F, vier G, neun H,
fünfundzwanzig I, einem J, einem K, zwei L, elf M, achtundzwanzig N, einem O, einem P, einem Q, sieben R,
dreizehn S, sieben T, sieben U, fünf V, vier W, einem X, einem Y, zehn Z, einem Ä, einem Ö, vier Ü und einem ß.}

\textsc{Dieser Satz besteht aus acht A, sechs B, sechs C, sieben D, fünfundvierzig E, acht F, vier G, neun H,
fünfundzwanzig I, einem J, einem K, zwei L, elf M, achtundzwanzig N, einem O, einem P, einem Q, sieben R,
dreizehn S, sieben T, sieben U, fünf V, vier W, einem X, einem Y, zehn Z, einem Ä, einem Ö, vier Ü und einem ß.}

\textup{Dieser Satz besteht aus acht A, sechs B, sechs C, sieben D, fünfundvierzig E, acht F, vier G, neun H,
fünfundzwanzig I, einem J, einem K, zwei L, elf M, achtundzwanzig N, einem O, einem P, einem Q, sieben R,
dreizehn S, sieben T, sieben U, fünf V, vier W, einem X, einem Y, zehn Z, einem Ä, einem Ö, vier Ü und einem ß.}

\textit{Dieser Satz besteht aus acht A, sechs B, sechs C, sieben D, fünfundvierzig E, acht F, vier G, neun H,
fünfundzwanzig I, einem J, einem K, zwei L, elf M, achtundzwanzig N, einem O, einem P, einem Q, sieben R,
dreizehn S, sieben T, sieben U, fünf V, vier W, einem X, einem Y, zehn Z, einem Ä, einem Ö, vier Ü und einem ß.}

\textsw{Dieser Satz besteht aus acht A, sechs B, sechs C, sieben D, fünfundvierzig E, acht F, vier G, neun H,
fünfundzwanzig I, einem J, einem K, zwei L, elf M, achtundzwanzig N, einem O, einem P, einem Q, sieben R,
dreizehn S, sieben T, sieben U, fünf V, vier W, einem X, einem Y, zehn Z, einem Ä, einem Ö, vier Ü und einem ß.}

%\text{Dieser Satz besteht aus \textbf{acht A}, \textit{sechs B}, \textsc{sechs C}, \texttt{sieben D}, \textrm{fünfundvierzig E}, acht F, vier G, neun H,
%fünfundzwanzig I, einem J, einem K, zwei L, elf M, achtundzwanzig N, einem O, einem P, einem Q, sieben R,
%dreizehn S, sieben T, sieben U, fünf V, vier W, einem X, einem Y, zehn Z, einem Ä, einem Ö, vier Ü und einem ß.}

{
\fontsize{22pt}{24pt}\selectfont\hfill
The end is never the end is never the end is never the end is never
the end is never the end is never the end is never the end is never
the end is never the end is never the end is never the end is never
the end is never the end is never the end is never the end is never
the end is never the end is never the end is never the end is never
the end is never the end is never the end is never the end is never
the end is never the end is never the end is never the end is never
the end is never the end is never the end is never the end is never
the end is never the end is never the end is never the end is never
the end is never the end is never the end is never the end \par
}

{
\fontsize{22pt}{24pt}\selectfont \textbf{Ich bin eine H1 Überschrift}
}

\end{document}